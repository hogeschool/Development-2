\section{Course program}
	The course is structured into five lectures. The lectures take place during the six weeks of the course, but are not necessarily in a one-to-one correspondance with the course weeks.

	\subsection{Lecture 1 - data structures}
		This lecture covers basic concepts of data structures as a means to avoid brittle representation of data by means of multiple basic variables:

		\paragraph*{Topics}
			\begin{itemize}
				\item Mechanism of abstraction;
				\item The necessity for data structures;
				\item Data structures in Python (class);
				\item Semantics (Heap, Stack);
				\item Layers of abstraction.
			\end{itemize}

			\subsection{Lecture 2 - lists}
				This lecture covers a well-known data structure that exemplifies good design and reasoning in terms of encapsulation and genericity:

				\paragraph*{Topics}
					\begin{itemize}
						\item The need for a variable to contain an \textit{unknown} number of values;
						\item Abstraction of list: \texttt{Node (Head), Tail, Empty};
						\item Implementation of list (Python 3);
						\item Semantics of list: \texttt{Heap and Stack}
					\end{itemize}
				
			\subsection{Lecture 3 - functions}
				This lecture covers abstraction over (groups of) instructions and statements through functions (partly recap):

				\paragraph*{Topics}
				\begin{itemize}
					\item Abstraction operations (functions)
					\item The need for functions;
					\item Creating and using functions in Python;
					\item Formal and actual parameters and return;
					\item Brief introduction to: scope (local and global variables) and visibility;
					\item Syntax and semantics;
					\item Introduction to recursion;
				\end{itemize}
				

			\subsection{Lecture 4 - higher order functions and SQL}
				This lecture covers higher order functions (HOF's) and connects them with the world of databases by sketching the connection between list HOF's and SQL queries:

				\paragraph*{Topics}
					\begin{itemize}
						\item What are HOF's and \textit{why we do need them}?
						\item Functions as parameter;
						\item Lambda: $\lambda$-expressions (syntax and semantics);
						\item Fundamental operations on list: \texttt{transform, filter, fold};
						\item Using HOF's;
						\item SQL vs list HOF's.
					\end{itemize}

			\subsection{Lecture 5 - methods}
				This lecture covers abstraction of data structures and functions within the single container of classes:

				\paragraph*{Topics}
					\begin{itemize}
						\item Joining functions (methods) and data to classes;
						\item Designing a class;
						\item Concrete implementation of a class;
						\item Syntax and semantics;
						\item special method names;
						\item rebuilding the list data structure;
						\item Brief introduction immutability and mutability.
					\end{itemize}

			\subsection{Lecture 6 - collections library (optional)}
				This lecture covers the existing collections library of Python, and illustrates how it can be used instead of rebuilt by hand:

				\paragraph*{Topics}
					\begin{itemize}
						\item lists;
						\item tuples;
						\item maps;
						\item sets.
					\end{itemize}
